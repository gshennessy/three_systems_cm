%% Beginning of file 'sample701.tex'
%%
%% Version 7.0.1. Created May 2025.
%% Version 7. Created January 2025.  
%%
%% AASTeX v7+ calls the following external packages:
%% times, hyperref, ifthen, hyphens, longtable, xcolor, 
%% bookmarks, array, rotating, ulem, and lineno 
%%
%% RevTeX is no longer used in AASTeX v7+.
%%
\documentclass[linenumbers,trackchanges]{aastex701}
%%
%% This initial command takes arguments that can be used to easily modify 
%% the output of the compiled manuscript. Any combination of arguments can be 
%% invoked like this:
%%
%% \documentclass[argument1,argument2,argument3,...]{aastex7}
%%
%% Six of the arguments are typestting options. They are:
%%
%%  twocolumn   : two text columns, 10 point font, single spaced article.
%%                This is the most compact and represent the final published
%%                derived PDF copy of the accepted manuscript from the publisher
%%  default     : one text column, 10 point font, single spaced (default).
%%  manuscript  : one text column, 12 point font, double spaced article.
%%  preprint    : one text column, 12 point font, single spaced article.  
%%  preprint2   : two text columns, 12 point font, single spaced article.
%%  modern      : a stylish, single text column, 12 point font, article with
%% 		  wider left and right margins. This uses the Daniel
%% 		  Foreman-Mackey and David Hogg design.
%%
%% Note that you can submit to the AAS Journals in any of these 6 styles.
%%
%% There are other optional arguments one can invoke to allow other stylistic
%% actions. The available options are:
%%
%%   astrosymb    : Loads Astrosymb font and define \astrocommands. 
%%   tighten      : Makes baselineskip slightly smaller, only works with 
%%                  the twocolumn substyle.
%%   times        : uses times font instead of the default.
%%   linenumbers  : turn on linenumbering. Note this is mandatory for AAS
%%                  Journal submissions and revisions.
%%   trackchanges : Shows added text in bold.
%%   longauthor   : Do not use the more compressed footnote style (default) for 
%%                  the author/collaboration/affiliations. Instead print all
%%                  affiliation information after each name. Creates a much 
%%                  longer author list but may be desirable for short 
%%                  author papers.
%% twocolappendix : make 2 column appendix.
%%   anonymous    : Do not show the authors, affiliations, acknowledgments,
%%                  and author contributions for dual anonymous review.
%%  resetfootnote : Reset footnotes to 1 in the body of the manuscript.
%%                  Useful when there are a lot of authors and affiliations
%%		    in the front matter.
%%   longbib      : Print article titles in the references. This option
%% 		    is mandatory for PSJ manuscripts.
%%
%% Since v6, AASTeX has included \hyperref support. While we have built in 
%% specific %% defaults into the classfile you can manually override them 
%% with the \hypersetup command. For example,
%%
%% \hypersetup{linkcolor=red,citecolor=green,filecolor=cyan,urlcolor=magenta}
%%
%% will change the color of the internal links to red, the links to the
%% bibliography to green, the file links to cyan, and the external links to
%% magenta. Additional information on \hyperref options can be found here:
%% https://www.tug.org/applications/hyperref/manual.html#x1-40003
%%
%% The "bookmarks" has been changed to "true" in hyperref
%% to improve the accessibility of the compiled pdf file.
%%
%% If you want to create your own macros, you can do so
%% using \newcommand. Your macros should appear before
%% the \begin{document} command.
%%
\newcommand{\vdag}{(v)^\dagger}
\newcommand\aastex{AAS\TeX}
\newcommand\latex{La\TeX}
%%%%%%%%%%%%%%%%%%%%%%%%%%%%%%%%%%%%%%%%%%%%%%%%%%%%%%%%%%%%%%%%%%%%%%%%%%%%%%%%
%%
%% The following section outlines numerous optional output that
%% can be displayed in the front matter or as running meta-data.
%%
%% Running header information. A short title on odd pages and 
%% short author list on even pages. Note that this
%% information may be modified in production.
%%\shorttitle{AASTeX v7.0.1 Sample article}
%%\shortauthors{The Terra Mater collaboration}
%%
%% Include dates for submitted, revised, and accepted.
%%\received{February 1, 2025}
%%\revised{March 1, 2025}
%%\accepted{\today}
%%
%% Indicate AAS Journal the manuscript was submitted to.
%%\submitjournal{PSJ}
%% Note that this command adds "Submitted to " the argument.
%%
%% You can add a light gray and diagonal water-mark to the first page 
%% with this command:
\watermark{draft}
%% where "text", e.g. DRAFT, is the text to appear.  If the text is 
%% long you can control the water-mark size with:
%% \setwatermarkfontsize{dimension}
%% where dimension is any recognized LaTeX dimension, e.g. pt, in, etc.
%%%%%%%%%%%%%%%%%%%%%%%%%%%%%%%%%%%%%%%%%%%%%%%%%%%%%%%%%%%%%%%%%%%%%%%%%%%%%%%%
%%
%% Use this command to indicate a subdirectory where figures are located.
%%\graphicspath{{./}{figures/}}
%% This is the end of the preamble.  Indicate the beginning of the
%% manuscript itself with \begin{document}.

\begin{document}

\title{Astrometric parameters for three binary systems expressed in the ICRF}

%% A significant change from AASTeX v6+ is in the author blocks. Now an email
%% address is required for each author. This means that each author requires
%% at least one of the following:
%%
%% \author
%% \affiliation
%% \email
%%
%% If these three commands are not available for each author, the latex
%% compiler will issue an error and if you force the latex compiler to continue,
%% it will generate an incomplete pdf.
%%
%% Multiple \affiliation commands are allowed and authors can also include
%% an optional \altaffiliation to indicate a status, i.e. Hubble Fellow. 
%% while affiliations are indexed as footnotes, altaffiliations are noted with
%% with a non-numeric footnote that is set away from the numeric \affiliation 
%% footnotes. NOTE that if an \altaffiliation command is used it must 
%% come BEFORE the \affiliation call, right after the \author command, in 
%% order to place the footnotes in the proper location. Because non-numeric
%% symbols are used, \altaffiliation should be used sparingly.
%%
%% In v7+ the \author command takes an optional argument which provides 
%% additional metadata about the author. Authors can provide the 16 digit 
%% ORCID, the surname (family or last) name, the given (first or fore-) name, 
%% and a name suffix, e.g. "Jr.". The syntax is:
%%
%% \author[orcid=0000-0002-9072-1121,gname=Gregory,sname=Schwarz]{Greg Schwarz}
%%
%% This name metadata in not shown, it is only for parsing by the peer review
%% system so authors can be more easily identified. This name information will
%% also be sent to the publisher so they can include it in the CROSSREF 
%% metadata. Including an orcid will hyperlink the author name to the 
%% author's ORCID page. Note that  during compilation, LaTeX will do some 
%% limited checking of the format of the ID to make sure it is valid. If 
%% the "orcid-ID.png" image file is  present or in the LaTeX pathway, the 
%% ORCID icon will appear next to the authors name.
%%
%% Even though emails are now required for each author, the \email does not
%% produce output in the compiled manuscript unless the optional "show" command
%% is used. For example,
%%
%% \email[show]{greg.schwarz@aas.org}
%%
%% All "shown" emails are show in the bottom left of the first page. Due to
%% space constraints, only a few emails should be shown. 
%%
%% To identify a corresponding author, use the \correspondingauthor command.
%% The command appends "Corresponding Author: " to the argument it appears at
%% the bottom left of the first page like the output from \email. 

\author[orcid=0000-0001-8017-5115,gname=Gregory,sname=Hennessy]{Gregory S. Hennessy}
\affiliation{US Naval Observatory}
\email[show]{greg.hennessy@cox.net}  

\author[orcid=0000-0002-2768-1606,gname=Julien, sname=Frouard]{Julien Frouard} 
\affiliation{US Naval Observatory}
\email[show]{julien.h.frouard.civ@us.navy.mil}

\author[gname=Rachel,sname=Matson]{Rachel Matson}
\affiliation{US Naval Observatory}
\email[show]{rachel.a.matson2.civ@us.navy.mil}

\author[gname=Brian,sname=Mason]{Brian Mason}
\affiliation{US Naval Observatory}
\email[show]{brian.d.mason.civ@us.navy.mil}

\author[gname=George,sname=Kaplan]{George Kaplan}
\affiliation{US Naval Observatory}
\email[show]{gk@gkaplan.us}

%% Use the \collaboration command to identify collaborations. This command
%% takes an optional argument that is either a number or the word "all"
%% which tells the compiler how many of the authors above the command to
%% show. For example "\collaboration[all]{(DELVE Collaboration)}" wil include
%% all the authors above this command.
%%
%% Mark off the abstract in the ``abstract'' environment. 
\begin{abstract}

  We present astrometric parameters describing the center-of-mass (CM) motions of three nearby binary systems: Sirius
(HIP~32349), Procyon (HIP~37279), and $\mu$~Cassiopeiae (HIP~5336). When combined with well-determined orbital elements,
these CM astrometric parameters enable accurate predictions of the motions of these systems in the International
Celestial Reference Frame (ICRF). This approach yields substantially improved positional predictions compared to those
obtained by propagating Hipparcos catalog positions alone, with direct relevance for high-precision celestial
navigation. 
\end{abstract}

%% Keywords should appear after the \end{abstract} command. 
%% The AAS Journals now uses Unified Astronomy Thesaurus (UAT) concepts:
%% https://astrothesaurus.org
%% You will be asked to selected these concepts during the submission process
%% but this old "keyword" functionality is maintained in case authors want
%% to include these concepts in their preprints.
%%
%% You can use the \uat command to link your UAT concepts back its source.
\keywords{\uat{Binary Stars}{154} --- \uat{Stellar astronomy}{1583}  --- \uat{Proper Motions}{1295}}

%% From the front matter, we move on to the body of the paper.
%% Sections are demarcated by \section and \subsection, respectively.
%% Observe the use of the LaTeX \label
%% command after the \subsection to give a symbolic KEY to the
%% subsection for cross-referencing in a \ref command.
%% You can use LaTeX's \ref and \label commands to keep track of
%% cross-references to sections, equations, tables, and figures.
%% That way, if you change the order of any elements, LaTeX will
%% automatically renumber them.

\section{Introduction} \label{sec:intro}

Accurate astrometric positions of bright stars are essential for navigation applications, both on the ground and in
space. Although any enumeration of astrometric catalogs is necessarily incomplete, major catalogs produced over recent
decades include the Fifth Fundamental Catalogue (FK5; \citealt{1988VeARI..32....1F}), Hipparcos
\citep{1997A&A...323L..49P}, and the three Gaia data releases
\citep{2016A&A...595A...2G,2018A&A...616A...1G,2023A&A...674A...1G}. 

All astrometric catalogs are subject to systematic and statistical limitations. In particular, stellar multiplicity has
long been recognized as a significant source of degraded catalog accuracy
\citep{1975ARA&A..13..295V,1997A&A...323L..49P,1999A&A...341..121S,2001A&A...372..935P,2017ApJ...840L...1M}. An
additional complication arises from the markedly different temporal sampling of the two principal space-based
astrometric missions: Hipparcos observed for approximately three years, while Gaia DR3 spans roughly 33 months, with the
two missions separated by nearly 24 years. This highly non-uniform time coverage complicates the direct combination of
Hipparcos and Gaia astrometric data \citep{2019A&A...623A..72K}. 

During a routine update of the USNO Guidance and Navigation Catalog (Version~1.2; \citealt{USNOGNC}), it became apparent
that accurate propagated positions for several well-known bright stars---including Sirius, Procyon, $\alpha$~Centauri,
and Polaris---could not be obtained using standard five-parameter astrometric solutions alone. These stars share two
critical properties: extreme brightness, which results in incomplete representation of the primary components in Gaia
DR3, and membership in binary or multiple systems. 

Sirius is among the earliest stellar systems for which an orbital solution was determined
\citep{10.1093/mnras/49.8.420a}, with a highly precise modern orbit derived from Hubble Space Telescope (HST)
observations \citep{2017ApJ...840...70B}. Its white dwarf companion, Sirius~B, was discovered by Alvan Clark
\citep{Clark1862} and is approximately 10 magnitudes fainter than the primary in the visible band. The system has a
semimajor axis of $7.4957 \pm 0.0025\arcsec$ and an orbital period of $50.1284 \pm 0.0043$~yr
\citep{2017ApJ...840...70B}. 

The Procyon system \citep{Auwers1862} consists of a subgiant primary and a white dwarf companion differing in visual
magnitude by approximately 10.5 \citep{Wenger2000SIMBAD}. The orbit has a semimajor axis of $4.3075 \pm 0.0016\arcsec$
and a period of $40.840 \pm 0.022$~yr \citep{2015ApJ...813..106B}. 

$\mu$~Cassiopeiae is a binary system composed of two main-sequence stars with visual magnitudes of 5.14 and 11.45
\citep{Wenger2000SIMBAD}. The orbit has a semimajor axis of $0.9985 \pm 0.0013\arcsec$ and a period of $21.568 \pm
0.015$~yr \citep{2020ApJ...904..112B}. For all three systems, propagating positions without accounting for orbital
motion introduces positional errors comparable to the orbital semimajor axis. 

\section{Data and Data Reduction} \label{sec:data_dr}

All three systems analyzed in this study---Sirius, Procyon, and $\mu$~Cassiopeiae---have been observed with the Hubble
Space Telescope, with the relevant astrometric analyses presented in three studies led by Bond
\citep{2017ApJ...840...70B,2015ApJ...813..106B,2020ApJ...904..112B}. In each case, the analyses relied on measurements
of relative position angles and separations between the binary components, rather than absolute right ascension and
declination. Consequently, the results are insensitive to telescope pointing errors, provided that both components
remain well within the detector field of view. 

We retrieved archival imaging data from the Mikulski Archive for Space Telescopes (MAST), including observations beyond
those obtained under proposal~12296. Inspection of the images revealed that the WFC3 observations provide substantially
higher signal-to-noise ratios than those obtained with WFPC2. For a linear astrometric model fitted to uniformly sampled
position measurements with constant single-epoch precision, the formal uncertainty of the proper motion scales as
$T^{-3/2}$, where $T$ is the total observational time baseline
\citep{Wielen1997,vanLeeuwen2007,Lindegren2021}. Accordingly, we restrict our analysis to only the WFC3 data when deriving
astrometric parameters. 

We processed each of the three stars independently. We inspected each data file to ensure
that we could measure the position of the primary and secondary. In many, if not most cases, the secondary was of low
enough flux that a fitted position could not be performed, in this case the pixel value of the peak flux was
chosen, and the error was set to 1 pixel. With respect to the primary, when the peak was not saturated, a position was
fitted. For these two situations, a two dimensional gaussian was fit using a Astropy Levenberg-Marquardt algorithm and
least squares statistic, and the fitted error was used. When the primary was saturated, two points on each of the four
diffraction spikes were chosen, and then the intersection x,y locations were calculated using Cramer's
rule\citep{Cramer1750}. The x,y positions were converted into Ra, Dec pairs using the Astropy WCS routines, being
careful to take into account the 1 based WCS routines and the 0 base used in python. We list the ICRF positions, along
with the calculated errors in Table~\ref{tab:positions}. 

Once the ICRF positions were calculated as a function of time, the center of mass of the three systems can be
calculated at each observation date, along with the astrometric parameters for the center of mass of the system. The
initial starting point for each calculation was the Hipparcos position of the primary star of each system, using the
orbital parameters from Table~\ref{tab:orbits}. The separation and position angle was calculated at the Hipparcos epoch
of 1991.25, and the masses of the two stars in order to get the ICRF position of the center of mass of the system. For
each HST position in Table~\ref{tab:positions}, the effect of the finite parallax was removed from both the primary and
secondary's position, the center of mass position calculated, and then the proper motions fit. The proper motion in RA
was fit using $RA*\cos(Dec)$ as the current standard in astrometry\citep{1997A&A...323L..49P}. The value of the
declination in the fit was the Declination of the system CM at epoch 1991.25. The calculated values are listed in
Table~\ref{tab:cm_astrometry}.

We provide tabulated positions, in increments of 5 years, from 2025 to 2050 in Table~\ref{tab:cm_ephemeris}. The table
was designed to resemble Table 12 of \citet{2021AJ....162...14A}. We only tablulated ever fifth year since we did not
include the effect of parallax in our ephemeris to save space, which should be a simple exercise if the  users needs
this in the future. 

\begin{deluxetable}{ccccccccc}
\tablehead{\colhead{System Name} & \colhead{Dataset} & \colhead{Time} & \colhead{Pri RA} & \colhead{Pri Dec} &
  \colhead{Sec RA} & \colhead{Sec Dec} & \colhead{$\sigma$ RA} & \colhead{$\sigma$ Dec}} 
\tablecaption{ICRF positions and errors for three non--single systems \label{tab:positions}}
\tabletypesize{\scriptsize}
\tablewidth{468pt} 
\startdata
$\mu$ Cas & ib7j02010 & 2010-01-09 & 17.08474076 & 54.91585401 & 17.08529206 & 54.91611363 & 0.036 & 0.037 \\
$\mu$ Cas & ib7j02020 & 2010-01-09 & 17.08474075 & 54.91585400 & 17.08529205 & 54.91611362 & 0.036 & 0.037 \\
$\mu$ Cas & ib7j02030 & 2010-01-09 & 17.08474067 & 54.91585394 & 17.08529197 & 54.91611356 & 0.036 & 0.037 \\
$\mu$ Cas & ib7j02040 & 2010-01-09 & 17.08474023 & 54.91585382 & 17.08529153 & 54.91611344 & 0.036 & 0.037 \\
$\mu$ Cas & ibk702010 & 2010-12-03 & 17.08628543 & 54.91545152 & 17.08681730 & 54.91572010 & 0.006 & 0.007 \\
$\mu$ Cas & ibk702020 & 2010-12-03 & 17.08628541 & 54.91545152 & 17.08681728 & 54.91572010 & 0.006 & 0.007 \\
$\mu$ Cas & ibk702030 & 2010-12-03 & 17.08628531 & 54.91545153 & 17.08681718 & 54.91572011 & 0.006 & 0.007 \\
$\mu$ Cas & ibk702040 & 2010-12-03 & 17.08628529 & 54.91545153 & 17.08681716 & 54.91572011 & 0.006 & 0.007 \\
$\mu$ Cas & ibti02010 & 2011-12-05 & 17.08772929 & 54.91509275 & 17.08822680 & 54.91536490 & 0.025 & 0.025 \\
$\mu$ Cas & ibti02020 & 2011-12-05 & 17.08772945 & 54.91509258 & 17.08822696 & 54.91536473 & 0.025 & 0.025 \\
$\mu$ Cas & ibti02030 & 2011-12-05 & 17.08772964 & 54.91509266 & 17.08822715 & 54.91536481 & 0.025 & 0.025 \\
$\mu$ Cas & ibti02040 & 2011-12-05 & 17.08772962 & 54.91509266 & 17.08822713 & 54.91536481 & 0.025 & 0.025 \\
$\mu$ Cas & ic1k02010 & 2012-12-02 & 17.08948331 & 54.91464474 & 17.08993252 & 54.91491257 & 0.008 & 0.008 \\
$\mu$ Cas & ic1k02020 & 2012-12-02 & 17.08948336 & 54.91464470 & 17.08993257 & 54.91491253 & 0.008 & 0.008 \\
$\mu$ Cas & ic1k02030 & 2012-12-02 & 17.08975353 & 54.91452513 & 17.09020274 & 54.91479296 & 0.008 & 0.008 \\
$\mu$ Cas & ic1k02040 & 2012-12-02 & 17.08948346 & 54.91464469 & 17.08993267 & 54.91491252 & 0.008 & 0.008 \\
$\mu$ Cas & ica102010 & 2013-10-25 & 17.09112169 & 54.91411555 & 17.09151404 & 54.91437157 & 0.053 & 0.053 \\
$\mu$ Cas & ica102020 & 2013-10-25 & 17.09112169 & 54.91411555 & 17.09151404 & 54.91437157 & 0.053 & 0.053 \\
$\mu$ Cas & ica102030 & 2013-10-25 & 17.09112187 & 54.91411545 & 17.09151422 & 54.91437147 & 0.053 & 0.053 \\
$\mu$ Cas & ica102040 & 2013-10-25 & 17.09112187 & 54.91411545 & 17.09151422 & 54.91437147 & 0.053 & 0.053 \\
$\mu$ Cas & icjx02010 & 2015-01-06 & 17.09318699 & 54.91371843 & 17.09348098 & 54.91394378 & 0.023 & 0.023 \\
$\mu$ Cas & icjx02020 & 2015-01-06 & 17.09305488 & 54.91364079 & 17.09334887 & 54.91386614 & 0.023 & 0.023 \\
$\mu$ Cas & icjx02030 & 2015-01-06 & 17.09318698 & 54.91371839 & 17.09348097 & 54.91394374 & 0.023 & 0.023 \\
$\mu$ Cas & icjx02040 & 2015-01-06 & 17.09318698 & 54.91371839 & 17.09348097 & 54.91394374 & 0.023 & 0.023 \\
$\mu$ Cas & icvd02010 & 2016-07-11 & 17.09566986 & 54.91290773 & 17.09579716 & 54.91306039 & 0.036 & 0.036 \\
$\mu$ Cas & icvd02020 & 2016-07-11 & 17.09566792 & 54.91290556 & 17.09579522 & 54.91305822 & 0.036 & 0.036 \\
$\mu$ Cas & icvd02030 & 2016-07-11 & 17.09566792 & 54.91290661 & 17.09579522 & 54.91305927 & 0.036 & 0.036 \\
$\mu$ Cas & icvd02040 & 2016-07-11 & 17.09566791 & 54.91290640 & 17.09579521 & 54.91305906 & 0.036 & 0.036 \\
$\mu$ Cas & icvd02050 & 2016-07-11 & 17.09567078 & 54.91290497 & 17.09579808 & 54.91305763 & 0.036 & 0.036 \\
$\mu$ Cas & icvd02060 & 2016-07-11 & 17.09566817 & 54.91290360 & 17.09579547 & 54.91305626 & 0.036 & 0.036 \\
Sirius & ibk703010 & 2010-09-02 & 101.28474474 & -16.71961117 & 101.28737234 & -16.71957535 & 0.020 & 0.020 \\
Sirius & ibk703020 & 2010-09-02 & 101.28474503 & -16.71961085 & 101.28737234 & -16.71957531 & 0.020 & 0.020 \\
Sirius & ibk703030 & 2010-09-02 & 101.28474534 & -16.71961014 & 101.28737194 & -16.71957361 & 0.020 & 0.020 \\
Sirius & ibk703040 & 2010-09-02 & 101.28474540 & -16.71961020 & 101.28737216 & -16.71957374 & 0.020 & 0.020 \\
Sirius & ibti03010 & 2011-10-01 & 101.28455076 & -16.72005425 & 101.28729072 & -16.71987776 & 0.019 & 0.019 \\
Sirius & ibti03020 & 2011-10-01 & 101.28455091 & -16.72005394 & 101.28729065 & -16.71987732 & 0.019 & 0.019 \\
Sirius & ibti03030 & 2011-10-01 & 101.28454961 & -16.72005414 & 101.28729036 & -16.71987771 & 0.019 & 0.019 \\
Sirius & ibti03040 & 2011-10-01 & 101.28454908 & -16.72005462 & 101.28729048 & -16.71987780 & 0.019 & 0.019 \\
Sirius & ic1k03010 & 2012-09-26 & 101.28436754 & -16.72042249 & 101.28717661 & -16.72011666 & 0.029 & 0.028 \\
Sirius & ic1k03020 & 2012-09-26 & 101.28436799 & -16.72042175 & 101.28717651 & -16.72011655 & 0.029 & 0.028 \\
Sirius & ic1k03030 & 2012-09-26 & 101.28436664 & -16.72042145 & 101.28717603 & -16.72011632 & 0.029 & 0.028 \\
Sirius & ic1k03040 & 2012-09-26 & 101.28436645 & -16.72042121 & 101.28717609 & -16.72011586 & 0.029 & 0.028 \\
Sirius & ica103010 & 2014-03-31 & 101.28387978 & -16.72101333 & 101.28677062 & -16.72052229 & 0.016 & 0.016 \\
Sirius & ica103020 & 2014-03-31 & 101.28387972 & -16.72101302 & 101.28677057 & -16.72052236 & 0.016 & 0.016 \\
Sirius & ica103030 & 2014-03-31 & 101.28387918 & -16.72101404 & 101.28677045 & -16.72052262 & 0.016 & 0.016 \\
Sirius & ica103040 & 2014-03-31 & 101.28387896 & -16.72101410 & 101.28677075 & -16.72052314 & 0.016 & 0.016 \\
Sirius & icvd03010 & 2016-08-20 & 101.28366750 & -16.72186767 & 101.28666418 & -16.72108281 & 0.013 & 0.013 \\
Sirius & icvd03020 & 2016-08-20 & 101.28366781 & -16.72186745 & 101.28666409 & -16.72108238 & 0.013 & 0.013 \\
Sirius & icvd03030 & 2016-08-20 & 101.28366800 & -16.72186614 & 101.28666417 & -16.72108100 & 0.013 & 0.013 \\
Sirius & icvd03040 & 2016-08-20 & 101.28366782 & -16.72186719 & 101.28666438 & -16.72108175 & 0.013 & 0.013 \\
Procyon & ibk701010 & 2011-02-07 & 114.82353177 & 5.22211914 & 114.82289759 & 5.22175284 & 0.015 & 0.014 \\
Procyon & ibk701020 & 2011-02-07 & 114.82353303 & 5.22212100 & 114.82289898 & 5.22175504 & 0.015 & 0.014 \\
Procyon & ibti01010 & 2012-03-09 & 114.82341201 & 5.22160704 & 114.82259165 & 5.22143779 & 0.014 & 0.014 \\
Procyon & ibti01020 & 2012-03-09 & 114.82341141 & 5.22160433 & 114.82259122 & 5.22143346 & 0.014 & 0.014 \\
Procyon & ic1k01010 & 2013-02-03 & 114.82323891 & 5.22145217 & 114.82231080 & 5.22143916 & 0.015 & 0.014 \\
Procyon & ic1k01020 & 2013-02-03 & 114.82323878 & 5.22145186 & 114.82231053 & 5.22144024 & 0.015 & 0.014 \\
Procyon & ica101010 & 2014-09-14 & 114.82288220 & 5.22084403 & 114.82186590 & 5.22112401 & 0.011 & 0.011 \\
Procyon & ica101020 & 2014-09-14 & 114.82288289 & 5.22084294 & 114.82186585 & 5.22112090 & 0.011 & 0.011 \\
Procyon & icvd01020 & 2016-10-03 & 114.82249139 & 5.22018343 & 114.82148783 & 5.22080915 & 0.013 & 0.013 \\
Procyon & icvd01040 & 2016-10-03 & 114.82249652 & 5.22018271 & 114.82148815 & 5.22080977 & 0.003 & 0.003
\enddata
\tablecomments{The RA and Dec positions are in decimal degrees, the errors in the RA and Dec are in arcseconds.}
\end{deluxetable}


\begin{deluxetable}{lcccccc}
\tablecaption{Orbital Elements for Sirius, Procyon, and $\mu$~Cassiopeiae\label{tab:orbits}}
\tablehead{
\colhead{Parameter} &
\colhead{Sirius} & \colhead{$\sigma$} &
\colhead{Procyon} & \colhead{$\sigma$} &
\colhead{$\mu$~Cas} & \colhead{$\sigma$}
}
\startdata
Orbital period, $P$ (yr)                 & 50.1284   & 0.0043  & 40.84    & 0.022   & 21.568    & 0.015  \\
Semimajor axis, $a$ (arcsec)             & 7.4957    & 0.0025  & 4.3075   & 0.0016  & 0.9985    & 0.0013 \\
Inclination, $i$ (deg)                   & 136.336   & 0.04    & 31.408   & 0.05    & 110.671   & 0.064 \\
Position angle of node, $\Omega$ (deg)   & 45.4      & 0.071   & 100.683  & 0.095   & 223.868   & 0.064 \\
Epoch of periastron, $T_0$ (yr)          & 1994.5715 & 0.0058  & 1968.076 & 0.023   & 1997.2235 & 0.0067 \\
Eccentricity, $e$                        & 0.59142   & 0.00037 & 0.39785  & 0.00025 & 0.5885    & 0.0011 \\
Longitude of periastron, $\omega$ (deg)  & 149.161   & 0.075   & 89.23    & 0.11    & 330.37    & 0.18   \\
Primary mass, $M_{\mathrm{A}}$ ($M_\odot$)   & 2.063     & 0.023   & 1.478    & 0.012   & 0.744     & 0.0122 \\
Secondary mass, $M_{\mathrm{B}}$ ($M_\odot$) & 1.018     & 0.011   & 0.592    & 0.006   & 0.1728    & 0.0035 \\
Systemic RV (km\,s$^{-1}$)                & -8.47     & \nodata & -4.115   & \nodata & -97       & \nodata \\
Parallax (milliarcsec)                   & 379.21    & 1.58    & 285.93   & 1.26    & 132.4     & 0.43 \\ 
\enddata
\tablecomments{Quoted uncertainties are $1\sigma$. The data are from the papers by Bond.}
\end{deluxetable}

\begin{deluxetable}{lcccccc}
\tablecaption{Center-of-Mass Astrometric Parameters at Epoch J1991.25\label{tab:cm_astrometry}}
\tablehead{
\colhead{Parameter} &
\colhead{Sirius CM} & \colhead{$\sigma$} &
\colhead{Procyon CM} & \colhead{$\sigma$} &
\colhead{$\mu$~Cas CM} & \colhead{$\sigma$}
}
\startdata
Right ascension, $\alpha$ (deg)              & 101.2884749 & \nodata & 114.8274028 & \nodata & 17.0539235 & \nodata \\
Declination, $\delta$ (deg)                  & $-16.7128166$ & \nodata & 5.2278806 & \nodata & 54.9242783 & \nodata \\
Proper motion in RA, $\mu_{\alpha*}$ (mas\,yr$^{-1}$)
                                            & $-551.945$ & 2.153 & $-738.334$ & 14.734 & 5945.242 & 15.56 \\
Proper motion in Dec, $\mu_{\delta}$ (mas\,yr$^{-1}$)
                                            & $-1256.457$ & 4.124 & $-1064.934$ & 12.269 & $-1607.441$ & 11.088 \\
Parallax, $\varpi$ (mas)                     & 379.21 & \nodata & 285.93 & \nodata & 132.4 & \nodata \\
Systemic radial velocity (km\,s$^{-1}$)      & $-8.47$ & \nodata & $-4.115$ & \nodata & $-97$ & \nodata \\
\enddata
\tablecomments{
Proper motions are given in the standard form $\mu_{\alpha*} = \dot{\alpha}\cos\delta$.
Quoted uncertainties are $1\sigma$.
}
\end{deluxetable}

\begin{deluxetable*}{lcrrrrrrrrrr}
\tablecaption{Predicted ephemeris for the Center of Mass, Primary, and Secondary for Sirius, Procyon, and
  $\mu$~Cassiopeiae\label{tab:cm_ephemeris}} 
\tablehead{
\colhead{Epoch} &
\colhead{System} &
\colhead{RA$_\mathrm{CM}$} &
\colhead{Dec$_\mathrm{CM}$} &
\colhead{RA$_A$} &
\colhead{Dec$_A$} &
\colhead{RA$_B$} &
\colhead{Dec$_B$} &
\colhead{$\Delta$RA} &
\colhead{$\Delta$Dec} &
\colhead{$\rho$} &
\colhead{PA} \\
\colhead{(yr)} &
\colhead{Name} &
\colhead{(deg)} &
\colhead{(deg)} &
\colhead{(deg)} &
\colhead{(deg)} &
\colhead{(deg)} &
\colhead{(deg)} &
\colhead{(as)} &
\colhead{(as)} &
\colhead{(deg)} &
\colhead{(as)}
}
\startdata
2025.00 & $\mu Cas$&  17.1508775 & 54.9091700 & 17.1508032 & 54.9091561 & 17.1511974 & 54.9092297&  -0.816 & -0.265 &  0.858&  71.998\\
2025.00 & Procyon  & 114.8204521 &  5.2178968 &114.8205303 &  5.2175014 &114.8202569 &  5.2188839&   0.980 & -4.977 &  5.072& 348.860\\
2025.00 & Sirius   & 101.2830719 &-16.7245958 &101.2821487 &-16.7251302 &101.2849427 &-16.7235128&  -9.633 & -5.823 & 11.256&  58.849\\
2030.00 & $\mu Cas$&  17.1652349 & 54.9069251 & 17.1651292 & 54.9068813 & 17.1656899 & 54.9071137&  -1.160 & -0.836 &  1.430&  54.219\\
2030.00 & Procyon  & 114.8194224 &  5.2164177 &114.8193282 &  5.2160207 &114.8196577 &  5.2174090&  -1.181 & -4.998 &  5.136&  13.297\\
2030.00 & Sirius   & 101.2822714 &-16.7263409 &101.2815204 &-16.7269673 &101.2837933 &-16.7250714&  -7.836 & -6.825 & 10.392&  48.944\\
2035.00 & $\mu Cas$&  17.1795907 & 54.9046786 & 17.1795119 & 54.9046292 & 17.1799301 & 54.9048913&  -0.866 & -0.944 &  1.281&  42.534\\
2035.00 & Procyon  & 114.8183927 &  5.2149386 &114.8181469 &  5.2146265 &114.8190064 &  5.2157178&  -3.082 & -3.929 &  4.993&  38.110\\
2035.00 & Sirius   & 101.2814708 &-16.7280859 &101.2810028 &-16.7287092 &101.2824193 &-16.7268228&  -4.883 & -6.791 &  8.364&  35.721\\
2040.00 & $\mu Cas$&  17.1939449 & 54.9024304 & 17.1939679 & 54.9024345 & 17.1938461 & 54.9024128&   0.252 &  0.078 &  0.264& 252.810\\
2040.00 & Procyon  & 114.8173630 &  5.2134595 &114.8170319 &  5.2133123 &114.8181897 &  5.2138270&  -4.151 & -1.853 &  4.546&  65.946\\
2040.00 & Sirius   & 101.2806703 &-16.7298310 &101.2806088 &-16.7302686 &101.2807949 &-16.7289441&  -0.642 & -4.768 &  4.811&   7.665\\
2045.00 & $\mu Cas$&  17.2082975 & 54.9001805 & 17.2082451 & 54.9001789 & 17.2085231 & 54.9001873&  -0.576 & -0.030 &  0.576&  86.986\\
2045.00 & Procyon  & 114.8163333 &  5.2119804 &114.8160653 &  5.2120523 &114.8170024 &  5.2118010&  -3.360 &  0.905 &  3.479& 105.072\\
2045.00 & Sirius   & 101.2798697 &-16.7315760 &101.2801192 &-16.7314409 &101.2793642 &-16.7318498&   2.603 &  1.472 &  2.990& 240.510\\
2050.00 & $\mu Cas$&  17.2226485 & 54.8979289 & 17.2225467 & 54.8978925 & 17.2230870 & 54.8980856&  -1.119 & -0.695 &  1.317&  58.141\\
2050.00 & Procyon  & 114.8153036 &  5.2105014 &114.8153591 &  5.2106692 &114.8151650 &  5.2100824&   0.696 &  2.112 &  2.224& 198.232\\
2050.00 & Sirius   & 101.2790692 &-16.7333211 &101.2788755 &-16.7329609 &101.2794616 &-16.7340510&  -2.021 &  3.924 &  4.414& 152.757\\
\enddata

\tablecomments{
RA offsets are $\Delta$RA$=\mathrm{RA}_B-\mathrm{RA}_A$ (including the $\cos\delta$ factor).
Position angles are measured east of north.}
\end{deluxetable*}


%\subsection{Figures\label{subsec:figures}}
%
%Figures subsection

\section{Conclusions} \label{sec:conclusions}

We provide center of mass astrometric parameters for three binary systems that are used in celestial navigation. We
provide an ephemeris for the center of mass, the primary, and the secondary on the ICRF for every fifth year from 2025
to 2050. 

\section{Revision tracking and color highlighting} \label{sec:highlight}

Revision tracking section. Not yet used. May be deleted if not useful.

%% Please use the acknowledgment and contribution environments. This will 
%% be anonomyized when the "anonymous" style option is used. 
\begin{acknowledgments}

This research is based on observations made with the NASA/ESA Hubble Space Telescope obtained from the Space Telescope Science
Institute, which is operated by the Association of Universities for Research in Astronomy, Inc., under NASA contract NAS
5–26555. These observations are associated with program 12296.

The data described here may be obtained from the MAST archive at
\dataset[doi:10.17909/T9RP4V]{https://dx.doi.org/10.17909/T9RP4V}.

\end{acknowledgments}

\begin{contribution}
%%This section gives authors the space to recognize author
%%contributions. The text inside this environment is NOT counted
%%towards the total word quanta. At a minimum, manuscripts are
%%expected to include this text: 

  Author Hennessy obtained the data, processed the data, wrote and edited the manuscript.

  Authors Frouard, Matson, Mason, and Kaplan provided subject matter expertise, code to propagate orbits, and editing of
  the manuscript.

  Not sure this is helpful and/or needed.
  
%% But authors are expected to provide more specific details, e.g. 
%%
%%SC was responsible for writing and submitting the manuscript.
%%WWM came up with the initial research concept and edited the manuscript.
%%OTS obtained the funding and edited the manuscript.
%%EBF provided the formal analysis and validation. He also edited the manuscript.
%%GEH Supervised the undergraduates, wrote the software and administers the project github and Zenodo repositories.
%%
%% Authors can use the Contributor Role Taxonomy (CRediT) at
%% https://credit.niso.org
%% for ideas on how write a good statement tailored to their needs.

\end{contribution}

%% To help institutions obtain information on the effectiveness of their 
%% telescopes the AAS Journals has created a group of keywords for telescope 
%% facilities.
%
%% Following the acknowledgments section, use the following syntax and the
%% \facility{} or \facilities{} macros to list the keywords of facilities used 
%% in the research for the paper.  Each keyword is check against the master 
%% list during copy editing.  Individual instruments can be provided in 
%% parentheses, after the keyword, but they are not verified.
\facilities{HST(WFPC2)}

%% Similar to \facility{}, there is the optional \software command to allow 
%% authors a place to specify which programs were used during the creation of 
%% the manuscript. Authors should list each code and include either a
%% citation or url to the code inside ()s when available.
\software{astropy\citep{2013A&A...558A..33A,2018AJ....156..123A,2022ApJ...935..167A},
  novas\citep{2012ascl.soft02003K},
  scipy\citep{Virtanen2020SciPy}}

%% Appendix material should be preceded with a single \appendix command.
%% There should be a \section command for each appendix. Mark appendix
%% subsections with the same markup you use in the main body of the paper.
%%
%% Each Appendix (indicated with \section) will be lettered A, B, C, etc.
%% The equation counter will reset when it encounters the \appendix
%% command and will number appendix equations (A1), (A2), etc. The
%% Figure and Table counter will not reset.

%\appendix

%\section{Appendix information}

%Appendix section.

%% For this sample we use BibTeX plus aasjournalv7.bst to generate the
%% the bibliography. The sample7.bib file was populated from ADS. To
%% get the citations to show in the compiled file do the following:
%%
%% pdflatex sample7.tex
%% bibtext sample7
%% pdflatex sample7.tex
%% pdflatex sample7.tex

\bibliography{sample701}{}
\bibliographystyle{aasjournalv7}

%% This command is needed to show the entire author+affiliation list when
%% the collaboration and author truncation commands are used.  It has to
%% go at the end of the manuscript.
%\allauthors

%% Include this line if you are using the \added, \replaced, \deleted
%% commands to see a summary list of all changes at the end of the article.
%\listofchanges

\end{document}

% End of file `sample7.tex'.
